\begin{recipe}
	[
	preparationtime = {\unit[30]{min}},
	bakingtime={\unit[20]{min}},
	bakingtemperature={\protect\bakingtemperature{fanoven=\unit[180]{°C}}},
	portion = {\portion{3}},
	calory,
	source
	]
	{Enchilada de Pollo}
	\graph
	{
		small=pictures/enchillada/enchillada.jpg
	}
	
	\ingredients
	{
		\unit[500]{g} & Pouletbrust \\
		6 & Tortillas \\
		2 & Knoblauchzehen \\
		1 & Paprika \\
		1 & Chilischote \\
		\unit[500-600]{g} & passierte Tomaten \\
		\unit[4]{EL} & Creme Fraiche \\
		& Reibkäse \\
		& Salz, Pfeffer, Paprikapulver \\
		& Basilikum \\
	}
	
	\preparation
	{
		\step Poulet klein schneiden, mit Salz und Pfeffer w"urzen und in "Ol anbraten.
		\step Paprika, Knoblauch und Chilischote klein schneiden.
		\step Poulet aus der Pfanne nehmen und beiseite stellen und das Gem"use in der Pfanne anbraten.
		\step In der Zwischenzeit das Poulet von Hand auseinander zupfen.
		\step Dann das Fleisch wieder in die Pfanne geben, kurz verr"uhren und 100-200g passierte Tomaten dazu geben und w"ahrend 10 Minuten k"ocheln lassen.
		\step F"ur die Sauce eine Knoblauchzehe klein schneiden und in etwas "Ol anbraten. 400g passierte Tomaten dazu geben und 5 Minuten k"ocheln lassen.
		\step Sauce danach mit etwas Salz, Pfeffer und Basilikum abschmecken.
		\step Zum Fleisch kann nach 10 Minuten die Creme Fraiche hinzugegeben werden. Danach mit Salz, Pfeffer und Paprikapulver abschmecken.
		\step F"ullung auf die Tortillas geben, zusammenrollen und in eine Gratinform legen. Die Sauce dar"uber geben und das ganze mit Reibk"ase "uberdecken. Bei 180 Grad f"ur 20 Minuten "uberbacken.
	}
\end{recipe}