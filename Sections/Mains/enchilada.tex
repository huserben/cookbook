\begin{recipe}
	[
	preparationtime = {\unit[30]{min}},
	bakingtime={\unit[20]{min}},
	bakingtemperature={\protect\bakingtemperature{fanoven=\unit[180]{°C}}},
	portion = {\portion{3}},
	calory,
	source
	]
	{Enchilada de Pollo}
	\graph
	{
		small=pictures/enchillada/enchillada.jpg
	}
	
	\ingredients
	{
		\unit[500]{g} & Pouletbrust \\
		6 & Tortillas \\
		2 & Knoblauchzehen \\
		1 & Paprika \\
		1 & Chilischote \\
		\unit[500-600]{g} & passierte Tomaten \\
		\unit[4]{EL} & Crème fraîche \\
		& Reibkäse \\
		& Salz, Pfeffer, Paprikapulver \\
		& Basilikum \\
	}
	
	\preparation
	{
		\step Poulet klein schneiden, mit Salz und Pfeffer würzen und in Öl anbraten.
		\step Paprika, Knoblauch und Chilischote klein schneiden.
		\step Poulet aus der Pfanne nehmen und beiseite stellen und das Gemüse in der Pfanne anbraten.
		\step In der Zwischenzeit das Poulet von Hand auseinander zupfen.
		\step Dann das Fleisch wieder in die Pfanne geben, kurz verrühren und 100-200g passierte Tomaten dazu geben und während 10 Minuten köcheln lassen.
		\step Für die Sauce eine Knoblauchzehe klein schneiden und in etwas Öl anbraten. 400g passierte Tomaten dazu geben und 5 Minuten köcheln lassen.
		\step Sauce danach mit etwas Salz, Pfeffer und Basilikum abschmecken.
		\step Zum Fleisch kann nach 10 Minuten die Crème fraîche hinzugegeben werden. Danach mit Salz, Pfeffer und Paprikapulver abschmecken.
		\step Füllung auf die Tortillas geben, zusammenrollen und in eine Gratinform legen. Die Sauce darüber geben und das ganze mit Reibkäse überdecken. Bei 180 Grad für 20 Minuten überbacken.
	}
\end{recipe}