\begin{recipe}
	[
	preparationtime = {\unit[25]{min}},
	bakingtime = {\unit[25-30]{min}},
	bakingtemperature={\protect\bakingtemperature{fanoven=\unit[200]{°C}}},
	portion,
	calory,
	source
	]
	{Kürbis Tarte}
	
	\graph
	{
		big=pictures/kuerbistarte/kuerbistarte.png
	}
	
	\ingredients
	{
		\unit[300]{g} & Mehl \\
		\unit[150]{g} & kalte Butter \\
		1 & Ei \\
		\unit[2]{EL} & eiskaltes Wasser \\		
		\unit[600]{g} & Hokkaido Kürbis \\
		\unit[2]{EL} & Olivenöl \\
		\unit[750]{g} & Hackfleisch (gemischt) \\
		\unit[2-3]{EL} & Tomatenmark \\
		& Reibkäse (Cheddar, Gouda) \\
		& Salz, Pfeffer, Zucker \\
		& Thymian, Rosmarin
	}
	
	\preparation
	{
		\step Für den Teig Mehl, Butter in Stückchen, Ei, 1 Prise Salz und Wasser verrühren, dann mit den Händen zu einem glatten Teig verkneten.
		\step Mit den Fingern in eine Tarteform drücken.
		\step Kürbis waschen, vierteln, entkernen und Fruchtfleisch würfeln.
		\step Öl erhitzen, Hack darin krümelig anbraten. Salz, Pfeffer und Kräuter dazugeben.
		\step Das Tomatenmark unterrühren und kurz anschwitzen. Mit Wasser ablöschen. Mit Zucker abschmecken und kurz köcheln lassen.
		\step Zuerst eine Schicht Kürbis auf den Boden legen, dann Hack darüber verteilen. Restlichen Kürbis darüber geben. Mit Käse bestreuen.
		\step Bei 200 Grad für 25-30 Minuten backen.
	}
	\hint
	{
		Man kann den Mürbeteig auch fertig kaufen.
	}
\end{recipe}