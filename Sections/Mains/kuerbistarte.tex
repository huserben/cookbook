\begin{recipe}
	[
	preparationtime = {\unit[25]{min}},
	bakingtime = {\unit[25-30]{min}},
	bakingtemperature={\protect\bakingtemperature{fanoven=\unit[200]{°C}}},
	portion,
	calory,
	source
	]
	{K"urbis Tarte}
	
	\graph
	{
		big=pictures/kuerbistarte/kuerbistarte.png
	}
	
	\ingredients
	{
		\unit[300]{g} & Mehl \\
		\unit[150]{g} & kalte Butter \\
		1 & Ei \\
		\unit[2]{EL} & eiskaltes Wasser \\		
		\unit[600]{g} & Hokkaidok"urbis \\
		\unit[2]{EL} & Oliven"ol \\
		\unit[750]{g} & Hackfleisch (gemischt) \\
		\unit[2-3]{EL} & Tomatenmark \\
		& Reibk"ase (Cheddar, Gouda) \\
		& Salz, Pfeffer, Zucker \\
		& Thymian, Rosmarin
	}
	
	\preparation
	{
		\step F"ur den Teig Mehl, Butter in St"uckchen, Ei, 1 Prise Salz und Wasser verr"uhren, dann mit den H"anden zu einem glatten Teig verkneten.
		\step Mit den Fingern in eine Tarteform dr"ucken.
		\step K"urbis waschen, vierteln, entkernen und Fruchtfleisch w"urfeln.
		\step "Ol erhitzen, Hack darin kr"umelig anbraten. Salz, Pfeffer und Kr"auter dazugeben.
		\step Tomatenmark unterr"uhren und kurz anschwitzen. Mit Wasser abl"oschen. Mit Zucker abschmecken und kurz k"ocheln lassen.
		\step Zuerst eine Schicht K"urbis auf den Boden legen, dann Hack dar"uber verteilen. Restlichen K"urbis dar"uber geben. Mit K"ase bestreuen.
		\step Bei 200 Grad f"ur 25-30 Minuten backen.
	}
	\hint
	{
		Man kann den M"urbeteig auch fertig kaufen.
	}
\end{recipe}