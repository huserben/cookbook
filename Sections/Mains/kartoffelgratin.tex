\begin{recipe}
	[
	preparationtime = {\unit[20]{min}},
	bakingtime={\unit[60]{min}},
	bakingtemperature={\protect\bakingtemperature{fanoven=\unit[180]{°C}}},
	portion = {\portion{4}},
	calory,
	source
	]
	{Kartoffelgratin}
	\graph
	{
		small=pictures/kartoffelgratin/kartoffelgratin.jpg
	}
	
	\ingredients
	{
		\unit[800]{g} & mehlig kochende Kartoffeln \\
		\unit[3]{dl} & Rahm \\
		\unit[3]{dl} & Milch \\
		1-2 & Knoblauchzehen \\
		& Salz, Pfeffer, Muskatnuss \\
		\unit[100]{g} & Reibk"ase \\
	}
	
	\preparation
	{
		\step Kartoffeln waschen, sch"alen und in gleichm"assige Scheiben von 2-3 mm schneiden.
		\step Karotffelscheiben in Gratinform schichten.
		\step Milch, Rahm und nach belieben Knoblauch mischen und w"urzen.
		\step Guss "uber die Kartoffeln giessen. Die Kartoffeln sollten dabei ganz bedeckt sein.
		\step Auf der untersten Rille des auf \unit[180]{C} vorgeheizten Ofens ca. \unit[40-50]{min} backen.
		\step K"ase auf Gratin verteilen, Ofentemperatur auf \unit[220]{C} erh"ohen und in der oberen H"alfte des Ofens ca. \unit[10]{min} "uberbacken.
	}
	\hint
	{
		Falls festkochende Kartoffeln verwendet werden, kann die Fl"ussigkeit um 1-2 dl verringert werden.
	}
\end{recipe}