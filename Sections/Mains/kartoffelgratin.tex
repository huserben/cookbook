\begin{recipe}
	[
	preparationtime = {\unit[20]{min}},
	bakingtime={\unit[60]{min}},
	bakingtemperature={\protect\bakingtemperature{fanoven=\unit[180]{°C}}},
	portion = {\portion{4}},
	calory,
	source
	]
	{Kartoffelgratin}
	\graph
	{
		small=pictures/kartoffelgratin/kartoffelgratin.jpg
	}
	
	\ingredients
	{
		\unit[800]{g} & mehlig kochende Kartoffeln \\
		\unit[3]{dl} & Rahm \\
		\unit[3]{dl} & Milch \\
		1-2 & Knoblauchzehen \\
		& Salz, Pfeffer, Muskatnuss \\
		\unit[100]{g} & Reibkäse \\
	}
	
	\preparation
	{
		\step Kartoffeln waschen, schälen und in gleichmässige Scheiben von 2 bis 3 mm schneiden.
		\step Kartoffelscheiben in Gratinform schichten.
		\step Milch, Rahm und nach Belieben Knoblauch mischen und würzen.
		\step Guss über die Kartoffeln giessen. Die Kartoffeln sollten dabei ganz bedeckt sein.
		\step Auf der untersten Rille des auf \unit[180]{C} vorgeheizten Ofens ca. \unit[40-50]{min} backen.
		\step Käse auf Gratin verteilen, Ofentemperatur auf \unit[220]{C} erhöhen und in der oberen Hälfte des Ofens ca. \unit[10]{min} überbacken.
	}
	\hint
	{
		Falls festkochende Kartoffeln verwendet werden, kann die Flüssigkeit um 1-2 dl verringert werden.
	}
\end{recipe}