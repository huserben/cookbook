\begin{recipe}
	[
	preparationtime = {\unit[45]{min}},
	bakingtime={\unit[20]{min}},
	bakingtemperature={\protect\bakingtemperature{fanoven=\unit[220]{°C}}},
	portion = {\portion{4}},
	calory,
	source
	]
	{Bolognese-Gratin}
	\graph
	{
		big=pictures/bolognese-gratin/bolognese-gratin.jpg
	}
	
	\ingredients
	{
		\unit[250-300]{g} & Teigwaren \\
		\unit[500]{g} & Hackfleisch \\
		\unit[200]{g} & R"uebli, gerieben \\
		1 & Knoblauchzehe \\
		\unit[100]{g} & Tomatenp"ure \\
		\unit[1]{dl} & Rahm \\
		& Salz, Pfeffer, Paprika \\
		& Reibk"ase \\
	}
	
	\preparation
	{
		\step Teigwaren in reichlich Salzwasser al dente kochen, abgiessen. In die ausgebutterte Gratinform geben.
		\step Fleisch in der heissen Bratbutter portionenweise rundum kr"aftig anbraten, beiseite stellen.
		\step Gem"use, Zwiebel und Knoblauch in derselben Pfanne and"ampfen. Tomatenp"uree und Fleisch dazugeben, mischen. Mit Bouillon und Rahm abl"oschen, aufkochen, w"urzen.
		\step Zugedeckt bei kleiner Hitze 25-30 Minuten schmoren.
		\step Auf die Teigwaren in der Gratinform geben, nach Belieben mischen, flach streichen. Mit K"ase bestreuen.
		\step In der Mitte des auf 220°C vorgeheizten Ofens 15-20 Minuten gratinieren.
	}
\end{recipe}