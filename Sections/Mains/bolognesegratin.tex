\begin{recipe}
	[
	preparationtime = {\unit[45]{min}},
	bakingtime={\unit[20]{min}},
	bakingtemperature={\protect\bakingtemperature{fanoven=\unit[220]{°C}}},
	portion = {\portion{4}},
	calory,
	source
	]
	{Bolognese-Gratin}
	\graph
	{
		big=pictures/bolognese-gratin/bolognese-gratin.jpg
	}
	
	\ingredients
	{
		\unit[250-300]{g} & Teigwaren \\
		\unit[500]{g} & Hackfleisch \\
		\unit[200]{g} & Rüebli, gerieben \\
		1 & Knoblauchzehe \\
		\unit[100]{g} & Tomatenpüree \\
		\unit[1]{dl} & Rahm \\
		& Salz, Pfeffer, Paprika \\
		& Reibkäse \\
	}
	
	\preparation
	{
		\step Teigwaren in reichlich Salzwasser al dente kochen, abgiessen. In die ausgebutterte Gratinform geben.
		\step Fleisch in der heissen Bratbutter portionsweise rundum kräftig anbraten, beiseite stellen.
		\step Gemüse, Zwiebel und Knoblauch in derselben Pfanne andämpfen. Tomatenpüree und Fleisch dazugeben, mischen. Mit Bouillon und Rahm ablöschen, aufkochen, würzen.
		\step Zugedeckt bei kleiner Hitze 25-30 Minuten schmoren.
		\step Auf die Teigwaren in der Gratinform geben, nach Belieben mischen, flach streichen. Mit Käse bestreuen.
		\step In der Mitte des auf 220 °C vorgeheizten Ofens 15-20 Minuten gratinieren.
	}
\end{recipe}