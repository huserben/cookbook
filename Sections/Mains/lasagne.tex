\begin{recipe}
	[
	preparationtime = {\unit[40]{min}},
	bakingtime={\unit[20]{min}},
	bakingtemperature={\protect\bakingtemperature{fanoven=\unit[200]{°C}}},
	portion = {\portion{4}},
	calory,
	source
	]
	{Lasagne}
	\graph
	{
		small=pictures/lasagne/lasagne.png
	}
	
	\ingredients
	{
		\unit[600-700]{g} & Hackfleisch gemischt \\
		reichlich & Knoblauch \\
		1 & Zwiebel gehackt \\
		\unit[1-2]{Würfel} & gebundene Bratensauce \\
		\unit[2]{dl} & Rotwein \\
		\unit[1]{Büchse} & Pelati gehackt \\
		\unit[5]{dl} & Weisse Sauce \\
		\unit[1]{Pack} & Nudelteigblätter \\
		\unit[1]{Pack} & Parmesan
	}
	
	\preparation
	{
		\step Fleisch mit dem Zwiebeln und Knoblauch anbraten.
		\step Mit der Bratensauce sowie dem Rotwein ablöschen.
		\step Reichlich Würzen und die Pelati hinzugeben. Das Ganze mindestens \unit[30]{min} köcheln lassen.
		\step Am Ende der Kochzeit die Sauce auf Schärfe probieren. Nach eigenem Gusto nochmals würzen.
		\step Die Weisse Sauce aufkochen
		\step Eine erste Lage Nudelteigblätter in die ausgebutterte Gratinform legen. Darüber dann eine Schicht Fleisch. Dies solange bis kein Fleisch mehr übrig ist. Die oberste Schicht sind wieder Nudelteigblätter.
		\step Die Weisse Sauce über die Lasagne geben und danach mit Parmesan bestreuen
		\step Lasagne für ca. \unit[20]{min} in den auf \unit[200-220]{°C} vorgeheizten Ofen schieben.
	}
	
	\hint
	{
		Empfohlene Gewürze: Pfeffer, Paprika (scharf), Oregano \\
		Die Weisse Sauce gibt es im Beutel zu kaufen (sind exakt \unit[5]{dl}) \\
		Wer es käsig mag, kann den Parmesan auch auf alle Fleischschichten geben
	}
\end{recipe}