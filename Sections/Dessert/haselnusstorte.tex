\begin{recipe}
	[
	preparationtime = {\unit[60]{min}},
	bakingtime = {\unit[60]{min}},
	bakingtemperature={\protect\bakingtemperature{fanoven=\unit[180]{°C}}},
	portion,
	calory,
	source
	]
	{Haselnusstorte}
	
	\graph
	{
		small=pictures/haselnusstorte/haselnusstorte.jpg
	}
	
	\ingredients
	{
		\multicolumn{2}{c}{\textbf{Torte}}\\
		\unit[220]{g} & Zucker \\
		\unit[300]{g} & Haselnüsse (gemahlen) \\
		6 & Eier \\
		& Rum \\
		& Salz \\
		& Zwieback \\
		\multicolumn{2}{c}{\textbf{Füllung}}\\
		\unit[2.5]{dl} & Rahm \\		
		\unit[100]{g} & Haselnüsse (gemahlen) \\
		\unit[1]{Pack} & Kaffeepulver \\
		\unit[4-5]{EL} & Zucker
	}
	
	\preparation
	{
		\step Zucker und Eigelbe in einer Schüssel verrühren bis die Masse hell ist
		\step Ein bis zwei Esslöffel Rum darunter rühren
		\step Eiweisse mit einer Prise Salz steif schlagen
		\step 2 Esslöffel Zucker zum Eischnee geben und kurz weiterschlagen
		\step Zwieback zusammen mit 300g Haselnüssen lagenweise mit Eischnee zur Masse geben.
		\step Eine 24 cm Springform mit Backpapier auslegen und die Ränder einfetten
		\step Masse in Springform geben und für 20 Minuten in der unteren Hälfte bei 180° backen. Danach Temperatur auf 150° stellen und für weitere 40 Minuten im Ofen lassen.
		\step Die Torte herausnehmen und auskühlen lassen. Danach im Kühlschrank für 2-3 Tage aufbewahren, dann kann die Torte quer halbiert werden.
		\step Für die Füllung den Rahm steif schlagen, danach die Haselnüsse, Kaffeepulver und den Zucker zum Rahm mischen und auf die halbierte Torte geben
	}
	\hint
	{
		Torte kann im Kühlschrank bis zu 2 Wochen aufbewahrt werden. Mit Füllung hält sie noch 1-2 Tage!
	}
\end{recipe}