\begin{recipe}
	[
	preparationtime = {\unit[30]{h}},
	bakingtime = {\unit[50]{min}},
	bakingtemperature={\protect\bakingtemperature{fanoven=\unit[180]{°C}}},
	portion,
	calory,
	source
	]
	{Crunchy Cream Torte}
	
	\graph
	{
		small=pictures/crunchycream/crunchycream.png
	}
	
	\ingredients
	{
		\unit[175]{g} & Butter \\
		\unit[200]{g} & Crunchy Cream (Ovo oder anderes) \\
		\unit[150]{g} & Dunkle Kochschokolade \\
		\unit[150]{g} & Rohzucker \\
		6 & Eier \\
		\unit[125]{g} & gemahlene Haselnüsse \\
		\unit[40]{g} & gehackte Haselnüsse
	}
	
	\preparation
	{
		\step Den Backofen auf 180° Celsius Umluft vorheizen. Die Springform
		mit Butter einfetten und mit etwas Mehl bestreuen.
		\step Die Schokolade, Crunchy Cream, Rohrzucker und Butter in einem grossen Topf bei geringer Hitze unter gelegentlichem Umrühren schmelzen lassen
		\step Topf vom Herd nehmen und die gemahlenen Haselnüsse unterrühren. Dann die Eigelbe	hinzugeben und vermischen.
		\step Die Eiweisse steif schlagen, zur Schokoladenmasse hinzugeben und vorsichtig unterheben
		\step Die Masse in die vorbereitete Springform geben und mit den gehackten Haselnüssen bestreuen
		\step Anschliessend für 50 bis 60 Minuten backen, bis die Oberfläche kross, aber die Torte innen
		noch weich ist. Nach der halben Backzeit die Torte mit Alufolie	abdecken, damit sie nicht zu dunkel wird.
		\step Die Torte vollständig abkühlen lassen und anschliessend aus der Springform lösen
	}
	
	\hint
	{
		Die Torte schmeckt auch am nächsten Tag sehr saftig und kann bis zu drei Tagen im Kühlschrank aufbewahrt werden
	}
\end{recipe}