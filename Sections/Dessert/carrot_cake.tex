\begin{recipe}
	[
	preparationtime = {\unit[30]{min}},
	bakingtime={\unit[60]{min}},
	bakingtemperature={\protect\bakingtemperature{fanoven=\unit[180]{°C}}},
	portion = {\portion{1}},
	calory,
	source
	]
	{Aargauer Rüeblitorte}
	
	\graph
	{
		big=pictures/rueblitorte/rueblitorte.png
	}
	
	\ingredients
	{
		\bf{Torte} \\
		\unit[300]{g} & Zucker \\
		5 & Eier \\
		\unit[300]{g} & gemahlene Mandeln \\
		\unit[300]{g} & Rüebli, fein gerieben \\
		1 & unbehandelte Zitrone \\
		\unit[4]{EL} & Maizena \\		
		\unit[4]{TL} & Backpulver \\
		\unit[1/2]{TL} & Zimt \\		
		& Nelkenpulver \\	
		\unit[1]{Prise} & Salz 	\\
		\unit[3]{EL} & Aprikosenkonfitüre \\
		\bf{Glasur} \\
		1 & Eiweiss \\
		\unit[300]{g} & Puderzucker \\
		\unit[2]{EL} & Zitronensaft \\
		\unit[2]{EL} & Wasser
	}
	
	\preparation
	{
		\step Zucker und 5 Eigelb in einer Schüssel gut verrühren bis die Masse hell ist.
		\step Mandeln, Rüebli, Zitronenschale und 2 Esslöffel Zitronensaft zusammen mit Maizena, Backpulver, Zimt und Nelkenpulver beigeben und gut vermischen.
		\step 5 Eiweisse mit dem Salz steif schlagen.
		\step Eiweiss unter Torten Masse ziehen und in die gefettete Tortenform geben.
		\step Bei \unit[180]{°C} für \unit[60]{min} in der unteren Hälfte des Ofens backen. Herausnehmen, etwas abkühlen lassen.
		\step Aprikosenkonfitüre in einer Pfanne erwärmen und auf Torte streichen.
		\step Für die Glasur 1 Eiweiss in einer Schüssel schaumig schlagen. 
		\step Puderzucker, Zitronensaft und Wasser beigeben und gut verrühren.
		\step Glasur über die Torte geben. Mit einem Spachtel verstreichen und trocknen lassen.
	}
\end{recipe}