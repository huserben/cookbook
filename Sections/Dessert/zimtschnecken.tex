\begin{recipe}
	[
	preparationtime = {\unit[3]{h}},
	bakingtime = {\unit[30]{min}},
	bakingtemperature={\protect\bakingtemperature{fanoven=\unit[180]{°C}}},
	portion,
	calory,
	source
	]
	{Zimtschnecken}
	
	\graph
	{
		small=pictures/zimtschnecken/zimtschnecken.jpg
	}
	
	\ingredients
	{
		\unit[500]{g} & Mehl \\
		\unit[0.5]{EL} & Salz \\
		\unit[1.5]{EL} & Zimt \\
		\unit[50]{g} & Rohzucker \\
		1 Beutel & Trockenhefe \\
		\unit[60]{g} & Butter \\
		\unit[2.5]{dl} & Milch \\
		\multicolumn{2}{c}{\textbf{Füllung}}\\
		\unit[125]{g} & Butter \\
		\unit[125]{g} & Rohzucker \\
		\unit[2]{EL} & Zimt \\
		\unit[2]{EL} & Mehl \\
		\multicolumn{2}{c}{\textbf{Glasur}}\\
		\unit[100]{g} & Puderzucker	\\	
		\unit[1.5]{EL} & Milch \\
	}
	
	\preparation
	{
		\step Mehl und alle Zutaten bis und mit Hefe in einer Schüssel mischen
		\step Butter und Milch beigeben, mischen, zu einem weichen, glatten Teig kneten
		\step Zugedeckt bei Raumtemperatur ca. 2 Std. aufs Doppelte aufgehen lassen
		\step Für die Füllung, Butter, Zucker, Zimt und Mehl verrühren
		\step Je 1 EL Mehl und Rohzucker mischen, Teig darauf zu einem Rechteck auswallen
		\step Füllung darauf verteilen, dabei ringsum einen Rand von ca. 1 cm frei lassen
		\step Teig von der Längsseite her aufrollen, mit einem Brotmesser ohne Druck in ca. 4 cm breite Stücke schneiden, auf zwei mit Backpapier belegte Bleche legen
		\step Zimtschnecken in den kalten Ofen schieben, ca. 30 Min. bei 180 Grad (Heissluft) backen. Herausnehmen und etwas abkühlen lassen.
		\step Puderzucker und Milch verrühren, Schnecken damit bestreichen, auf einem Gitter auskühlen
	}
\end{recipe}