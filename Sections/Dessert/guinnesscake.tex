\begin{recipe}
	[
	preparationtime = {\unit[30]{min}},
	bakingtime = {\unit[70]{min}},
	bakingtemperature={\protect\bakingtemperature{fanoven=\unit[175]{°C}}},
	portion,
	calory,
	source
	]
	{Guinness-Schoko-Kuchen}
	
	\graph
	{
		big=pictures/guinnesskuchen/guinnesscake.jpg
	}
	
	\ingredients
	{
		\unit[250]{g} & Butter, weich \\
		\unit[3-4]{dl} & Guinness \\
		\unit[75]{g} & Kakao \\
		\unit[50]{g} & Zartbitterschokolade \\		
		\unit[300]{g} & brauner Zucker \\
		\unit[200]{g} & Crème fraîche \\
		2 & Eier \\
		\unit[1]{Flasche} & Vanillemark \\
		\unit[1]{Pack} & Natron \\
		Prise & Salz \\
		\multicolumn{2}{c}{\textbf{Topping}}\\
		\unit[300]{g} & Frischkäse \\
		\unit[150]{g} & Puderzucker \\
		\unit[125]{ml} & Rahm \\
	}
	
	\preparation
	{
		\step Guinness in einen Topf geben, Butter hinzufügen und langsam schmelzen.
		\step Parallel Schokolade schmelzen und dann zu Guinness und Butter geben.
		\step Zucker und Kakao dazu geben und gut verrühren.
		\step Eier in einer grossen Schüssel aufschlagen bis sie cremig hell sind. Crème fraîche unterrühren und weiter schlagen.
		\step Schoko-Guinness Masse nach und nach dazu geben, dabei ständig rühren.
		\step Zum Schluss Mehl, Salz und Natron mischen und in den Teig sieben.
		\step Nochmal gut verrühren und dann in eine Springform geben.
		\step Bei \unit[175]{°}C im Ofen für 60-70 Minuten backen.
		\step Herausnehmen und komplett auskühlen lassen.
		\step Für das Topping: Sahne und Frischkäse aufschlagen bis die Masse sehr fest ist. Dann nach und nach den Puderzucker dazugeben und weiterschlagen.
		\step Masse auf den Kuchen geben und gleichmässig verteilen.
	}
\end{recipe}