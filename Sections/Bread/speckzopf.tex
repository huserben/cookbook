\begin{recipe}
	[ 
	preparationtime = {\unit[140]{min}},
	bakingtime = {\unit[45]{min}},
	bakingtemperature={\protect\bakingtemperature{fanoven=\unit[200]{°C}}},
	portion = {\portion{1}},
	calory
	]
	{Speckzopf}
	
	\graph
	{
		big=pictures/speckzopf/speckzopf.png
	}
	
	\ingredients
	{
		\unit[500]{g} & Zopfmehl (oder zu gleichen Teilen Weiss- und Dinkelmehl)\\
		\unit[1.5]{TL}  & Salz \\
		wenig & Pfeffer  \\
		etwas & Rosmarin\\
		\unit[10]{g} & Hefe\\
		\unit[3-3.5]{dl} & Milch \\
		\unit[25]{g} & Butter \\
		\unit[175-200]{g} & Speckw"urfeli \\
		1 & Ei
	}
	
	\preparation
	{%
		\step Butter schmelzen lassen, abk"uhlen lassen.
		\step Mehl, Salz, Pfeffer und Rosmarin in einer Sch"ussel mischen und eine Mulde formen.
		\step Hefe mit etwas Milch aufl"osen und in die Mulde giessen.
		\step Restliche Milch und Butter beif"ugen, das Ganze zu einem geschmeidigen Teig kneten.
		\step Zugedeckt eine Stunde ruhen lassen.
		\step Speck in einer Bratpfanne anbraten und dann abk"uhlen lassen. Danach unter den Teig kneten.
		\step Teig zu gleich grossen Str"angen rollen und zu einem Zopf flechten. Auf einem Backpapier nochmals 30-60 Minuten aufgehen lassen.
		\step Mit einem Eigelb bepinseln.
		\step Bei \unit[200]{C} im unteren Teil des Ofens f"ur 35-45 Minuten backen.
	}
	
	
\end{recipe}