\begin{recipe}
	[
	preparationtime = {\unit[70]{min}},
	bakingtime={\unit[15-20]{min}},
	bakingtemperature={\protect\bakingtemperature{fanoven=\unit[180]{°C}}},
	portion = {\portion{10}},
	calory,
	source=https://www.sweet-family.de/rezept/suesse-brezeln
	]
	{Süsse Brezel}
	\graph
	{
		small=pictures/sweet_pretzels/sweet_pretzels.jpg
	}
	
	\ingredients
	{
		\unit[200]{ml} & Milch \\
		\unit[500]{g} & Mehl \\
		\unit[1]{Würfel} & frische Hefe \\
		\unit[100]{g} & Zucker \\
		\unit[40]{g} & Butter \\
		2 & Eier \\
		\unit[1/2]{TL} & Salz
	}
	
	\preparation
	{
		\step Die Hefe zerbröckeln und mit 6 Esslöffel Milch glattrühren 
		\step Mehl in eine Schüssel sieben, eine Mulde hineindrücken und die Hefe hineingiessen. 10 Minuten gehen lassen.
		\step Zucker, Butter, restliche Milch, Salz und 1 Ei zugeben und alles zu einem Teig rühren
		\step Zugedeckt mindestens 30 Minuten gehen lassen
		\step Teig nochmal durchkneten und in ca. 10 Portionen aufteilen
		\step Zu einer Rolle auswallen und zu einer Brezel formen. Enden gut festdrücken.
		\step Übriges Ei verquirlen und Brezel damit bepinseln
		\step Im auf \unit[180]{C} Grad vorgeheizten Ofen (Umluft) für \unit[15-20]{min} goldgelb backen.
	}

	\hint
	{
		Man kann die Brezel pur oder mit Toppings (Hagelzucker, Schokosauce etc.) essen
	}
\end{recipe}